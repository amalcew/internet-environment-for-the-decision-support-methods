
\chapter{Wstęp}

Rozwój internetu oraz technik przetwarzania dużych zbiorów danych spowodował wzrost zapotrzebowania przedsiębiorstw na aplikacje internetowe i mobilne operujące na olbrzymich zestawach danych, zdolnych serwować ich klientom najlepiej dopasowane produkty.  Rozpatrywana problematyka wymaga nie tylko pokaźnej mocy obliczeniowej, ale i inteligentnych algorytmów potrafiących dostosować się do przetwarzanych przez nie zbiorów danych i preferencji klientów. Posiadając te informacje, algorytmy powinny być w stanie zaprezentować użytkownikom końcowym rankingi odwzorowujące ich upodobania, na przykład "najstraszniejsze filmy i seriale z gatunku horrorów dla fanów dreszczyku emocji".

Nie tylko firmy i korporacje zmagają się jednak z tą problematyką. Z pewnością można założyć, że każdy użytkownik internetu spotkał się choć raz z problemem wyboru najlepszego produktu spośród setek dostępnych artykułów, czy porywającej książki z gatunku fantastyki.

W takich sytuacjach przydatne okazują się inteligentne systemy wspomagania decyzji, a w szczególności systemy wielokryterialnego wspomagania decyzji (w skrócie WWD), które specjalizują się we wspieraniu decyzji przy uwzględnieniu wielu kryteriów i dużej liczby analizowanych wariantów.

\section{Cel i założenia projektu}

W niniejszej pracy podjęto się próby utworzenia aplikacji internetowej do użytku w realnych zastosowaniach wspomagania decyzji, jak i do edukacji uniwersyteckiej. System posiada zaimplementowane metody WWD rozwiązujące różne problematyki wspomagania decyzji.  Głównym założeniem projektu jest efektywny i intuicyjny interfejs użytkownika, który skutecznie wspomaga proces analizy działania zaimplementowanych metod, jak i wspomaga operacje porównywania analizowanych przez system wariantów. Ważnym aspektem jest także przemyślana implementacja systemu, ułatwiająca dodawanie nowych metod w sposób modularny.

Zaprojektowany system nadaje się do użytku lokalnego, jak i do szeroko pojętego udostępniania na dedykowanych serwerach. Aplikacja wspiera autoryzację użytkowników, udostępnianie projektów w trybie przeglądania oraz edycji, operacje wyjścia/wejścia (import, eksport danych, generowanie raportów),  bogatą wizualizację działania metod.
\newpage
\section{Zakres projektu i zastosowane technologie}

Opisany w niniejszej pracy projekt został zrealizowany przy użyciu nowoczesnych technik i narzędzi wykorzystywanych w inżynierii oprogramowania. W poniższym podrozdziale przedstawione zostały niektóre z najważniejszych programów i technologii wykorzystywanych w czasie tworzenia systemu.

\subsection{Użyte technologie informatyczne}

\subsubsection{Architektura mikroserwisów, \textit{Docker}}

Wiele współczesnych systemów informatycznych bazuje na architekturze \textbf{mikroserwisów}. Struktura ta pozwala na odseparowanie warstw programu oraz zabezpieczenie systemu przed atakami. W niniejszym projekcie została zastosowana szczególna wersja architektury mikroserwisowej, zwaną \textbf{konteneryzacją}. Technologia ta wykorzystuje wirtualizację na poziomie systemu operacyjnego oraz system operacyjny Linux - te dwie cechy zapewniają wysoką wydajność, konfigurowalność, kompatybilność międzyplatformową i bezpieczeństwo tworzonego rozwiązania. 

Aplikacja została podzielona na [wstawić finalną liczbę kontenerów] kontenery, hostowane przy użyciu platformy \textit{Docker}:
\begin{itemize}
	\item Silnik z metodami WWD,
	\item Aplikacja internetowa,
	\item Silnik bazodanowy.
\end{itemize}

\subsubsection{Języki programowania, frameworki, bazy danych}

Konteneryzacja posiada również jeszcze jedną, ogromną zaletę - zastosowanie mikroserwisów pozwala na dowolność w wykorzystaniu przeróżnych języków programowania i bibliotek programistycznych. Na wymienionych powyżej kontenerach zostały użyte różne języki oraz frameworki, z których najważniejszymi są:
\begin{itemize}
	\item \textbf{\textit{PHP}} - język skryptowy, szeroko wykorzystywany w budowaniu stron internetowych i aplikacji webowych. W ramach języka użyte zostały:
	\begin{itemize}
		\item \textbf{\textit{Laravel}} - Framework implementujący model MVC (z ang. \textit{Model-View-Controller}),
		\item \textbf{\textit{Livewire}} - Full-stackowy framework Laravela, służący do budowy interfejsów użytkownika,
		\item \textbf{\textit{Filament}} - Biblioteka wzorców i komponentów, wykorzystująca framework \textit{Livewire}
	\end{itemize}
	
	\item \textbf{\textit{Kotlin}} - międzyplatformowy, statycznie typowany język programowania ogólnego przeznaczenia, stworzony i rozwijany przez firmę \textit{JetBrains}. Język ten działa na maszynie wirtualnej \textit{Javy}. 
	
	\item \textbf{\textit{Java}} - obiektowy, międzyplatformowy język programowania, aktualnie rozwijany przez korporację \textit{Oracle}. Głównym frameworkiem wykorzystywanym w czasie tworzenia projektu był:
	\begin{itemize}
		\item \textbf{\textit{Spring Boot}} - narzędzie umożliwiające tworzenie mikrousług minimalnym nakładem pracy wejścia i konfiguracji.
	\end{itemize}
\end{itemize}

Technologią niezbędną w każdym projekcie informatycznym, tworzonym na środowisko webowe, jest silnik bazodanowy. Z uwagi na silną zależność pomiędzy danymi oraz niezbędną międzyplatformowość, w ramach projektu użyty został silnik zarządzania relacyjnymi bazami danych \textit{PostgreSQL}. Silnik ten jest natywnie wspierany przez rodzinę systemów operacyjnych \textit{Linux} i pozwala na łatwe zarządzanie dużymi wolumenami składowanych danych.

\subsection{Środowisko deweloperskie}

Ze względu na skalę projektu oraz mnogość użytych technologii i języków programowania, bardzo ważnym narzędziem używanym w czasie tworzenia systemu było Zintegrowane Środowisko Programistyczne (ang. \textit{IDE - integrated development environment}). Najczęściej używanymi \textit{IDE} były programy oferowane przez firmę \textit{JetBrains}, lidera tego typu rozwiązań na rynku informatycznym. Szczególnie przydatnymi \textit{IDE} były \textbf{\textit{PhpStorm}} (użyte przy pisaniu w języku PHP), \textbf{\textit{IntelliJ IDEA}} (użyte przy pisaniu w języku Kotlin), \textbf{\textit{DataGrip}} (budowa i zarządzanie bazą danych)

\subsection{Komunikacja i praca równoległa}

\subsubsection{System kontroli wersji \textit{GIT}, platforma \textit{Github}}

Niechybnie najważniejszym narzędziem użytym w czasie implementacji systemu jest system kontroli wersji \textit{GIT}. Systemem kontroli wersji nazywamy narzędzie wspomagające proces tworzenia oprogramowania w sposób współbieżny przez wielu programistów jednocześnie. Narzędzie \textit{GIT} umożliwia również tworzenie historii zmian, przydatnej w przypadku implementacji nowych, eksperymentalnych modułów projektu. 

Kod źródłowy projektu umieszczony został w repozytorium platformy \textit{Github}. Platforma ta poza udostępnianiem kodu zawiera wiele narzędzi, pomocnych w czasie realizowania projektu.  Ważniejszymi z nich jest tablica zadań kanban, narzędzia automatycznego wdrażania zmian \textit{Github Actions}, czy \textit{Github Issues} umożliwiające planowanie zadań oraz recenzowanie zmian w kodzie.

\subsubsection{Tablica zadań \textit{Kanban}}

Metodyka wykorzystywana w inżynierii oprogramowania do zarządzania zasobami i zadaniami projektowymi. Metodę zaadaptowało wiele platform i narzędzi zarządzania projektami, w ramach projektu użyte zostało narzędzie \textit{Github Projects}, hostowane na platformie \textit{Github}.

\subsubsection{Platforma \textit{Discord}}

Chmurowa usługa internetowa pozwalająca na hostowanie serwerów głosowych oraz czatu, głównie wykorzystywana przez graczy oraz twórców treści. Discord używany jest również przez profesjonalne firmy programistyczne, jako komunikator internetowy.

\newpage

\section{Układ pracy}

TODO