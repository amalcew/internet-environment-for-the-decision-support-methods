
\chapter{Wstęp}

Rozwój internetu oraz technik przetwarzania dużych zbiorów danych spowodował wzrost zapotrzebowania przedsiębiorstw na aplikacje internetowe i mobilne operujące na olbrzymich zestawach danych, zdolnych serwować ich klientom najlepiej dopasowane produkty.  Rozpatrywana problematyka wymaga nie tylko pokaźnej mocy obliczeniowej, ale i inteligentnych algorytmów potrafiących dostosować się do przetwarzanych przez nie zbiorów danych i preferencji klientów. Posiadając te informacje, algorytmy powinny być w stanie zaprezentować użytkownikom końcowym rankingi odwzorowujące ich upodobania, na przykład "najstraszniejsze filmy i seriale z gatunku horrorów dla fanów dreszczyku emocji".

Nie tylko firmy i korporacje zmagają się jednak z tą problematyką. Z pewnością można założyć, że każdy użytkownik internetu spotkał się choć raz z problemem wyboru najlepszego produktu spośród setek dostępnych artykułów, czy porywającej książki z gatunku fantastyki.

W takich sytuacjach przydatne okazują się inteligentne systemy wspomagania decyzji, a w szczególności systemy wielokryterialnego wspomagania decyzji (w skrócie WWD), które specjalizują się we wspieraniu decyzji przy uwzględnieniu wielu kryteriów i dużej liczby analizowanych wariantów.

\section{Cel i założenia projektu}

W niniejszej pracy podjęto się próby utworzenia aplikacji internetowej do użytku w realnych zastosowaniach wspomagania decyzji, jak i do edukacji uniwersyteckiej. System posiada zaimplementowane metody WWD rozwiązujące różne problematyki wspomagania decyzji.  Głównym założeniem projektu jest efektywny i intuicyjny interfejs użytkownika, który skutecznie wspomaga proces analizy działania zaimplementowanych metod, jak i wspomaga operacje porównywania analizowanych przez system wariantów. Ważnym aspektem jest także przemyślana implementacja systemu, ułatwiająca dodawanie nowych metod w sposób modularny.

Zaprojektowany system nadaje się do użytku lokalnego, jak i do szeroko pojętego udostępniania na dedykowanych serwerach. Aplikacja wspiera autoryzację użytkowników, udostępnianie projektów w trybie przeglądania oraz edycji, operacje wyjścia/wejścia (import, eksport danych, generowanie raportów),  bogatą wizualizację działania metod.
\newpage
\section{Zakres projektu i zastosowane technologie}

Opisany w niniejszej pracy projekt został zrealizowany przy użyciu nowoczesnych technik i narzędzi wykorzystywanych w inżynierii oprogramowania. W poniższym podrozdziale przedstawione zostały niektóre z najważniejszych programów wykorzystywanych w czasie tworzenia systemu.

\subsection{Użyte technologie informatyczne}

TODO

\subsection{Środowisko deweloperskie \textit{JetBrains}}

TODO

\subsection{System kontroli wersji \textit{GIT}, platforma \textit{Github}}

Niechybnie najważniejszym narzędziem użytym w czasie implementacji systemu jest system kontroli wersji \textit{GIT}. Systemem kontroli wersji nazywamy narzędzie wspomagające proces tworzenia oprogramowania w sposób współbieżny przez wielu programistów jednocześnie. Narzędzie \textit{GIT} umożliwia również tworzenie historii zmian, przydatnej w przypadku implementacji nowych, eksperymentalnych modułów projektu. 

Kod źródłowy projektu umieszczony został w repozytorium platformy \textit{Github}. Platforma ta poza udostępnianiem kodu zawiera wiele narzędzi, pomocnych w czasie realizowania projektu.  Ważniejszymi z nich jest tablica zadań kanban, narzędzia automatycznego wdrażania zmian \textit{Github Actions}, czy \textit{Github Issues} umożliwiające planowanie zadań oraz recenzowanie zmian w kodzie.

\section{Układ pracy}

TODO